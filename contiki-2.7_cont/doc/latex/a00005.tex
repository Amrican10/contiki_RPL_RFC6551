\hypertarget{a00005}{\subsection{example-\/psock-\/server.\-c}
}

\begin{DoxyCodeInclude}
/*
 * This is a small example of how to write a TCP server using
 * Contiki's protosockets. It is a simple server that accepts one line
 * of text from the TCP connection, and echoes back the first 10 bytes
 * of the string, and then closes the connection.
 *
 * The server only handles one connection at a time.
 *
 */

#include <string.h>

/*
 * We include "contiki-net.h" to get all network definitions and
 * declarations.
 */
#include "contiki-net.h"

/*
 * We define one protosocket since we've decided to only handle one
 * connection at a time. If we want to be able to handle more than one
 * connection at a time, each parallell connection needs its own
 * protosocket.
 */
static struct psock ps;

/*
 * We must have somewhere to put incoming data, and we use a 10 byte
 * buffer for this purpose.
 */
static char buffer[10];

/*---------------------------------------------------------------------------*/
/*
 * A protosocket always requires a protothread. The protothread
 * contains the code that uses the protosocket. We define the
 * protothread here.
 */
static
PT_THREAD(handle_connection(struct psock *p))
{
  /*
   * A protosocket's protothread must start with a PSOCK_BEGIN(), with
   * the protosocket as argument.
   *
   * Remember that the same rules as for protothreads apply: do NOT
   * use local variables unless you are very sure what you are doing!
   * Local (stack) variables are not preserved when the protothread
   * blocks.
   */
  PSOCK_BEGIN(p);

  /*
   * We start by sending out a welcoming message. The message is sent
   * using the PSOCK_SEND_STR() function that sends a null-terminated
   * string.
   */
  PSOCK_SEND_STR(p, "Welcome, please type something and press return.\n");
  
  /*
   * Next, we use the PSOCK_READTO() function to read incoming data
   * from the TCP connection until we get a newline character. The
   * number of bytes that we actually keep is dependant of the length
   * of the input buffer that we use. Since we only have a 10 byte
   * buffer here (the buffer[] array), we can only remember the first
   * 10 bytes received. The rest of the line up to the newline simply
   * is discarded.
   */
  PSOCK_READTO(p, '\n');
  
  /*
   * And we send back the contents of the buffer. The PSOCK_DATALEN()
   * function provides us with the length of the data that we've
   * received. Note that this length will not be longer than the input
   * buffer we're using.
   */
  PSOCK_SEND_STR(p, "Got the following data: ");
  PSOCK_SEND(p, buffer, PSOCK_DATALEN(p));
  PSOCK_SEND_STR(p, "Good bye!\r\n");

  /*
   * We close the protosocket.
   */
  PSOCK_CLOSE(p);

  /*
   * And end the protosocket's protothread.
   */
  PSOCK_END(p);
}
/*---------------------------------------------------------------------------*/
/*
 * We declare the process.
 */
PROCESS(example_psock_server_process, "Example protosocket server");
/*---------------------------------------------------------------------------*/
/*
 * The definition of the process.
 */
PROCESS_THREAD(example_psock_server_process, ev, data)
{
  /*
   * The process begins here.
   */
  PROCESS_BEGIN();

  /*
   * We start with setting up a listening TCP port. Note how we're
   * using the UIP_HTONS() macro to convert the port number (1010) to
   * network byte order as required by the tcp_listen() function.
   */
  tcp_listen(UIP_HTONS(1010));

  /*
   * We loop for ever, accepting new connections.
   */
  while(1) {

    /*
     * We wait until we get the first TCP/IP event, which probably
     * comes because someone connected to us.
     */
    PROCESS_WAIT_EVENT_UNTIL(ev == tcpip_event);

    /*
     * If a peer connected with us, we'll initialize the protosocket
     * with PSOCK_INIT().
     */
    if(uip_connected()) {
      
      /*
       * The PSOCK_INIT() function initializes the protosocket and
       * binds the input buffer to the protosocket.
       */
      PSOCK_INIT(&ps, buffer, sizeof(buffer));

      /*
       * We loop until the connection is aborted, closed, or times out.
       */
      while(!(uip_aborted() || uip_closed() || uip_timedout())) {

        /*
         * We wait until we get a TCP/IP event. Remember that we
         * always need to wait for events inside a process, to let
         * other processes run while we are waiting.
         */
        PROCESS_WAIT_EVENT_UNTIL(ev == tcpip_event);

        /*
         * Here is where the real work is taking place: we call the
         * handle_connection() protothread that we defined above. This
         * protothread uses the protosocket to receive the data that
         * we want it to.
         */
        handle_connection(&ps);
      }
    }
  }
  
  /*
   * We must always declare the end of a process.
   */
  PROCESS_END();
}
/*---------------------------------------------------------------------------*/
\end{DoxyCodeInclude}
 