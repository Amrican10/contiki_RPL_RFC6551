\hypertarget{a00052}{\subsection{\-Protothreads}
\label{a00052}\index{\-Protothreads@{\-Protothreads}}
}


\-Protothreads are a type of lightweight stackless threads designed for severly memory constrained systems such as deeply embedded systems or sensor network nodes.  


\-Protothreads are a type of lightweight stackless threads designed for severly memory constrained systems such as deeply embedded systems or sensor network nodes. \-Protothreads provides linear code execution for event-\/driven systems implemented in \-C. \-Protothreads can be used with or without an \-R\-T\-O\-S.

\-Protothreads are a extremely lightweight, stackless type of threads that provides a blocking context on top of an event-\/driven system, without the overhead of per-\/thread stacks. \-The purpose of protothreads is to implement sequential flow of control without complex state machines or full multi-\/threading. \-Protothreads provides conditional blocking inside \-C functions.

\-The advantage of protothreads over a purely event-\/driven approach is that protothreads provides a sequential code structure that allows for blocking functions. \-In purely event-\/driven systems, blocking must be implemented by manually breaking the function into two pieces -\/ one for the piece of code before the blocking call and one for the code after the blocking call. \-This makes it hard to use control structures such as if() conditionals and while() loops.

\-The advantage of protothreads over ordinary threads is that a protothread do not require a separate stack. \-In memory constrained systems, the overhead of allocating multiple stacks can consume large amounts of the available memory. \-In contrast, each protothread only requires between two and twelve bytes of state, depending on the architecture.

\begin{DoxyNote}{\-Note}
\-Because protothreads do not save the stack context across a blocking call, {\bfseries local variables are not preserved when the protothread blocks}. \-This means that local variables should be used with utmost care -\/ {\bfseries if in doubt, do not use local variables inside a protothread!}
\end{DoxyNote}
\-Main features\-:


\begin{DoxyItemize}
\item \-No machine specific code -\/ the protothreads library is pure \-C
\end{DoxyItemize}


\begin{DoxyItemize}
\item \-Does not use error-\/prone functions such as longjmp()
\end{DoxyItemize}


\begin{DoxyItemize}
\item \-Very small \-R\-A\-M overhead -\/ only two bytes per protothread
\end{DoxyItemize}


\begin{DoxyItemize}
\item \-Can be used with or without an \-O\-S
\end{DoxyItemize}


\begin{DoxyItemize}
\item \-Provides blocking wait without full multi-\/threading or stack-\/switching
\end{DoxyItemize}

\-Examples applications\-:


\begin{DoxyItemize}
\item \-Memory constrained systems
\end{DoxyItemize}


\begin{DoxyItemize}
\item \-Event-\/driven protocol stacks
\end{DoxyItemize}


\begin{DoxyItemize}
\item \-Deeply embedded systems
\end{DoxyItemize}


\begin{DoxyItemize}
\item \-Sensor network nodes
\end{DoxyItemize}

\-The protothreads \-A\-P\-I consists of four basic operations\-: initialization\-: \-P\-T\-\_\-\-I\-N\-I\-T(), execution\-: \-P\-T\-\_\-\-B\-E\-G\-I\-N(), conditional blocking\-: \-P\-T\-\_\-\-W\-A\-I\-T\-\_\-\-U\-N\-T\-I\-L() and exit\-: \-P\-T\-\_\-\-E\-N\-D(). \-On top of these, two convenience functions are built\-: reversed condition blocking\-: \-P\-T\-\_\-\-W\-A\-I\-T\-\_\-\-W\-H\-I\-L\-E() and protothread blocking\-: \-P\-T\-\_\-\-W\-A\-I\-T\-\_\-\-T\-H\-R\-E\-A\-D().

\begin{DoxySeeAlso}{\-See also}
\hyperlink{a00052}{\-Protothreads \-A\-P\-I documentation}
\end{DoxySeeAlso}
\-The protothreads library is released under a \-B\-S\-D-\/style license that allows for both non-\/commercial and commercial usage. \-The only requirement is that credit is given.\hypertarget{a00052_authors}{}\subsubsection{\-Authors}\label{a00052_authors}
\-The protothreads library was written by \-Adam \-Dunkels $<$\href{mailto:adam@sics.se}{\tt adam@sics.\-se}$>$ with support from \-Oliver \-Schmidt $<$\href{mailto:ol.sc@web.de}{\tt ol.\-sc@web.\-de}$>$.\hypertarget{a00052_pt-desc}{}\subsubsection{\-Protothreads}\label{a00052_pt-desc}
\-Protothreads are a extremely lightweight, stackless threads that provides a blocking context on top of an event-\/driven system, without the overhead of per-\/thread stacks. \-The purpose of protothreads is to implement sequential flow of control without using complex state machines or full multi-\/threading. \-Protothreads provides conditional blocking inside a \-C function.

\-In memory constrained systems, such as deeply embedded systems, traditional multi-\/threading may have a too large memory overhead. \-In traditional multi-\/threading, each thread requires its own stack, that typically is over-\/provisioned. \-The stacks may use large parts of the available memory.

\-The main advantage of protothreads over ordinary threads is that protothreads are very lightweight\-: a protothread does not require its own stack. \-Rather, all protothreads run on the same stack and context switching is done by stack rewinding. \-This is advantageous in memory constrained systems, where a stack for a thread might use a large part of the available memory. \-A protothread only requires only two bytes of memory per protothread. \-Moreover, protothreads are implemented in pure \-C and do not require any machine-\/specific assembler code.

\-A protothread runs within a single \-C function and cannot span over other functions. \-A protothread may call normal \-C functions, but cannot block inside a called function. \-Blocking inside nested function calls is instead made by spawning a separate protothread for each potentially blocking function. \-The advantage of this approach is that blocking is explicit\-: the programmer knows exactly which functions that block that which functions the never blocks.

\-Protothreads are similar to asymmetric co-\/routines. \-The main difference is that co-\/routines uses a separate stack for each co-\/routine, whereas protothreads are stackless. \-The most similar mechanism to protothreads are \-Python generators. \-These are also stackless constructs, but have a different purpose. \-Protothreads provides blocking contexts inside a \-C function, whereas \-Python generators provide multiple exit points from a generator function.\hypertarget{a00052_pt-autovars}{}\subsubsection{\-Local variables}\label{a00052_pt-autovars}
\begin{DoxyNote}{\-Note}
\-Because protothreads do not save the stack context across a blocking call, local variables are not preserved when the protothread blocks. \-This means that local variables should be used with utmost care -\/ if in doubt, do not use local variables inside a protothread!
\end{DoxyNote}
\hypertarget{a00052_pt-scheduling}{}\subsubsection{\-Scheduling}\label{a00052_pt-scheduling}
\-A protothread is driven by repeated calls to the function in which the protothread is running. \-Each time the function is called, the protothread will run until it blocks or exits. \-Thus the scheduling of protothreads is done by the application that uses protothreads.\hypertarget{a00052_pt-impl}{}\subsubsection{\-Implementation}\label{a00052_pt-impl}
\-Protothreads are implemented using local continuations. \-A local continuation represents the current state of execution at a particular place in the program, but does not provide any call history or local variables. \-A local continuation can be set in a specific function to capture the state of the function. \-After a local continuation has been set can be resumed in order to restore the state of the function at the point where the local continuation was set.

\-Local continuations can be implemented in a variety of ways\-:


\begin{DoxyEnumerate}
\item by using machine specific assembler code,
\item by using standard \-C constructs, or
\item by using compiler extensions.
\end{DoxyEnumerate}

\-The first way works by saving and restoring the processor state, except for stack pointers, and requires between 16 and 32 bytes of memory per protothread. \-The exact amount of memory required depends on the architecture.

\-The standard \-C implementation requires only two bytes of state per protothread and utilizes the \-C switch() statement in a non-\/obvious way that is similar to \-Duff's device. \-This implementation does, however, impose a slight restriction to the code that uses protothreads in that the code cannot use switch() statements itself.

\-Certain compilers has \-C extensions that can be used to implement protothreads. \-G\-C\-C supports label pointers that can be used for this purpose. \-With this implementation, protothreads require 4 bytes of \-R\-A\-M per protothread. 