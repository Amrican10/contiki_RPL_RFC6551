\hypertarget{a00054}{\subsection{\-The \-Rime communication stack}
\label{a00054}\index{\-The Rime communication stack@{\-The Rime communication stack}}
}


\-The \-Rime communication stack provides a set of lightweight communication primitives ranging from best-\/effort anonymous local area broadcast to reliable network flooding.  


\-The \-Rime communication stack provides a set of lightweight communication primitives ranging from best-\/effort anonymous local area broadcast to reliable network flooding. \-The protocols in the \-Rime stack are arranged in a layered fashion, where the more complex protocols are implemented using the less complex protocols.

\-We have chosen the communication primitives in the \-Rime stack based on what typical sensor network protocols use. \-Applications or protocols running on top of the \-Rime stack attach at any layer of the stack and use any of the communication primitives.

\-The \-Rime stack supports both single-\/hop and multi-\/hop communication primitives. \-The multi-\/hop primitives do not specify how packets are routed through the network. \-Instead, as the packet is sent across the network, the application or upper layer protocol is invoked at every node to choose the next-\/hop neighbor. \-This makes it possible to implement arbitrary routing protocols on top of the multi-\/hop primitives.

\-Protocols or applications running on top of the \-Rime stack can implement additional protocols that are not in the \-Rime stack. \-If a protocol or application running on top of the \-Rime stack would need a communication primitive that is not currently in the \-Rime stack, the application or protocol can implement it directly on top of other communication primitive in the stack.

\-For more information, see\-:

\-Adam \-Dunkels, \-Fredrik �sterlind, and \-Zhitao \-He. \-An adaptive communication architecture for wireless sensor networks. \-In \-Proceedings of the \-Fifth \-A\-C\-M \-Conference on \-Networked \-Embedded \-Sensor \-Systems (\-Sen\-Sys 2007), \-Sydney, \-Australia, \-November 2007.

\href{http://www.sics.se/~adam/dunkels07adaptive.pdf}{\tt http\-://www.\-sics.\-se/$\sim$adam/dunkels07adaptive.\-pdf} \href{http://www.sics.se/~adam/slides/dunkels07adaptive.ppt}{\tt http\-://www.\-sics.\-se/$\sim$adam/slides/dunkels07adaptive.\-ppt} 